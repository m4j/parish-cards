\documentclass[10pt, letterpaper, oneside]{article}
\usepackage[hmargin={0.3in,0.3in},vmargin={0.49in,0.9in},
            %showframe,
            %headheight=1.2in,
            %headsep=15pt,
            verbose]{geometry}
\usepackage[utf8]{inputenc}
% \usepackage[T1]{fontenc}
\usepackage[russian,english]{babel}
\usepackage[normalem]{ulem}
\usepackage[inline]{enumitem}
\usepackage{amssymb}
\usepackage{adjustbox}
\usepackage[table]{xcolor}
\usepackage[yyyymmdd,hhmmss]{datetime}
\usepackage{eso-pic}
\usepackage{longtable}
\usepackage{multirow}
\usepackage{arydshln}
\usepackage{array}
\usepackage{bigdelim}

\usepackage{fontspec}
\setmainfont{Helvetica}
%\setmainfont{Arial Unicode MS}

\setlength{\parindent}{0pt}
\setlength{\tabcolsep}{3pt}

\title{}
\author{}

\newcommand{\mline}{\cline{2-6}}
\newcommand{\mdashline}{\cdashline{2-6}}
\newcommand{\negpar}[1][-1em]{%
  \ifvmode\else\par\fi
  {\parindent=#1\leavevmode}\ignorespaces
}
% Disable hyphenation, see https://tex.stackexchange.com/a/177179
\tolerance=1
\emergencystretch=\maxdimen
\hyphenpenalty=10000
\hbadness=10000

\begin{document}
\selectlanguage{english}
\arrayrulecolor{gray!70}
\renewcommand{\arraystretch}{1.25}

\BLOCK{set previous = namespace(payor = "",method="",identifier="",section="")}
{\renewcommand{\arraystretch}{1.5}
\begin{longtable}[l]{p{0.07in} p{2in} p{3.45in} p{0.4in} p{0.5in} p{0.5in}l}
    & \large Служба/Service & ~ & ~ & \multicolumn{2}{l}{\large Record \#} & \tabularnewline 
    & \multicolumn{4}{l}{\large Число/Date} & & \tabularnewline
\end{longtable}
}

\vspace{-3.3em}
\begin{longtable}[l]{p{0.07in} | >{\raggedright}p{1.75in} | >{\raggedright}p{4.1in} >{\raggedleft}p{0.1in}| >{\raggedleft}p{0.5in} | >{\raggedleft}p{0.5in} |@{}l}
    \cline{5-6}
    \multicolumn{4}{l}{} & \multicolumn{1}{|l|}{\small Signature} & \small Amount  & \tabularnewline
    \mline
    & \multicolumn{1}{l}{cash -- \hspace{0.5in} checks --} & \multicolumn{2}{r|}{\small Сбор на нужды храма/Collection for church needs} & & & \tabularnewline
    \mline
    & \multicolumn{1}{l}{cash -- \hspace{0.5in} checks --} & \multicolumn{2}{r|}{\small Сбор на Благотворительный Фонд/Collection for Benevolent Fund} & &  & \tabularnewline
    \mline
    & \multicolumn{1}{l}{cash -- \hspace{0.5in} checks --} & \multicolumn{2}{r|}{\small Сбор у креста/Collection by the cross\_\_\_\_\_\_\_\_\_\_\_\_\_\_\_\_\_\_\_\_\_\_\_} & &  & \tabularnewline
    \mline
    & \multicolumn{1}{l}{cash -- \hspace{0.5in} checks --} & \multicolumn{2}{r|}{\small Копилка/Collection box} & &  & \tabularnewline
    \mline
    & \multicolumn{1}{l}{cash -- \hspace{0.5in} checks --} & \multicolumn{2}{r|}{\small Свечной ящик/Candle stand} & &  & \tabularnewline
    \mline
\BLOCK{for row in data}
    \BLOCK{set payor = row["Payor"]}
    \BLOCK{set method = row["Method"]}
    \BLOCK{set identifier = row["Identifier"]}
    \BLOCK{set section = row["Section"]}
    \BLOCK{set sectionName = row["Section_Name"]}
    \BLOCK{set numberOfItems = row["NumberOfItems"]}
    \BLOCK{set amount = row["Amount"]}
    \BLOCK{set total = row["Total"]}
    \BLOCK{set shouldDrawDelims = (method != previous.method or identifier != previous.identifier) and numberOfItems != "1"}
    \BLOCK{if shouldDrawDelims}
        \BLOCK{set previous.method = method}
        \BLOCK{set previous.identifier = identifier}
    \BLOCK{endif}
    \BLOCK{if section != previous.section}
        \BLOCK{set previous.section = section}
        \BLOCK{set previous.payor = ""}
        \mline
        & \multicolumn{1}{l}{} & \multicolumn{2}{r|}{} & &  & \tabularnewline
        \mline
        & \multicolumn{3}{l|}{} & \multicolumn{1}{c|}{\small \VAR{sectionName}} & & \tabularnewline
    \BLOCK{endif}
    \BLOCK{if payor != previous.payor}
        \mline
    \BLOCK{endif}
    \BLOCK{if shouldDrawDelims}
        \BLOCK{if payor == previous.payor}
            \mdashline
        \BLOCK{endif}
        \ldelim\{{\VAR{numberOfItems}}{0pt}
    \BLOCK{endif}
    \BLOCK{if payor != previous.payor}
        \BLOCK{set previous.payor = payor}
        &
        \BLOCK{if numberOfItems != "1"}
            \multirow[t]{\VAR{numberOfItems}}{=}{\VAR{payor}}
        \BLOCK{else}
            \VAR{payor}
        \BLOCK{endif}
    \BLOCK{else}
        & 
    \BLOCK{endif}
    & \VAR{row["Purpose"]}
    \BLOCK{if method == ""}
        & ~
    \BLOCK{else}
        & \hspace*{0.3em} \small \VAR{method}
    \BLOCK{endif}
    & \texttt{\VAR{identifier}}
% https://tex.stackexchange.com/questions/436415/rdelim-braces-in-table-misaligned-due-to-setstretch
    & \VAR{amount} & \BLOCK{if shouldDrawDelims}\rdelim\}{\VAR{numberOfItems}}{*}[\VAR{total}]\rule{0pt}{11pt}\BLOCK{endif}
    \tabularnewline
\BLOCK{endfor}
    \mline
    & & & & & & \tabularnewline \mline
    & & & & & & \tabularnewline \mline
    & & & & & & \tabularnewline \mline
    & & & & & & \tabularnewline \mline
    & & & & & & \tabularnewline \mline
    & & & & & & \tabularnewline \mline
    & & & & & & \tabularnewline \mline
    & & & & & & \tabularnewline \mline
    & & & & & & \tabularnewline \mline
    & & & & & & \tabularnewline \mline
    & & & & & & \tabularnewline \mline
    & & & & & & \tabularnewline \mline
    & & & & & & \tabularnewline \mline
    & & & & & & \tabularnewline \mline
    & & & & & & \tabularnewline \mline
    & & & & & & \tabularnewline \mline
    & & & & & & \tabularnewline \mline
    & & & & & & \tabularnewline \mline
    & & & & & & \tabularnewline \mline
    & & & & & & \tabularnewline \mline
    & & & & & & \tabularnewline \mline
    & & & & & & \tabularnewline \mline
    & & & & & & \tabularnewline \mline
    & & & & & & \tabularnewline \mline
    & & & & & & \tabularnewline \mline
    & & & & & & \tabularnewline \mline
    & & & & & & \tabularnewline \mline
    & & & & & & \tabularnewline \mline
    & & & & & & \tabularnewline \mline
    & & & & & & \tabularnewline \mline
    & & & & & & \tabularnewline \mline
    & & & & & & \tabularnewline \mline
    & & & & & & \tabularnewline \mline
    & & & & & & \tabularnewline \mline
    & & & & & & \tabularnewline \mline
    & & & & & & \tabularnewline \mline
    & & & & & & \tabularnewline \mline
    & & & & & & \tabularnewline \mline
    & & & & & & \tabularnewline \mline
    & & & & & & \tabularnewline \mline
    & & & & & & \tabularnewline \mline
    & & & & & & \tabularnewline \mline
    & & & & & & \tabularnewline \mline
\end{longtable}

\end{document}

